There are several ways in which distributed systems can benefit from support from programming languages and vice versa. In this talk we outline 3 ongoing experiences: (1) Protocol types are a variant of session types targeted at verification of distributed systems software in the “middleware stack”. The challenge here consists in reconciling tolerance to partial failures with coordinated (re)actions, while steering clear from impossibilities. (2) AEON is an actor-based distributed programming language that supports reasoning and programming across multiple actors in an atomic fashion, and adds a second layer of programming for capturing elasticity behavior. AEON reconciles high performance and “transaction-like” semantics by streamlining execution along a directed acyclic graph induced by an original ownership-based referencing discipline. Finally, (3) SASSY is a datacenter network architecture that supports communication with very tightly bounded delays, thus enabling the implementation of coordination protocols for synchronous systems which are much less costly, complex, and error-prone than their asynchronous counterparts. SASSY achieves this without overconstraining the entire network by logically splitting it into (a) an synchronous slice for time-sensitive coordination and (b) an asynchronous slice for regular best-effort traffic.