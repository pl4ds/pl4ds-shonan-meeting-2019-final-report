Distributed applications are traditionally developed as separate modules, often in different languages, which react to events, like user input, and in turn produce new events for the other modules. Separation into components requires time-consuming integration. Manual implementation of communication forces programmers to deal with low-level details. The combination of the two results in obscure distributed data flows scattered among multiple modules, hindering reasoning about the system as a whole.
The ScalaLoci distributed programming language addresses these issues with a coherent model based on placement types that enables reasoning about distributed data flows, supporting multiple software architectures via dedicated language features and abstracting over low-level communication details and data conversions. As we show, ScalaLoci simplifies developing distributed systems, reduces error-prone communication code and favors early detection of bugs.
