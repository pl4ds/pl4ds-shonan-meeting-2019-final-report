Relational database systems (RDBMSs) are widely used to persistently store and maintain data items.  “Serious computation,” however, predominantly occurs outside of the database kernel: the persistent data is extracted and placed on some conventional programming language’s heap (provided that the heap is able to hold the items at all), then processed, before the results are injected back into the database backend. This established approach exhibits a plethora of problems, among these (1) the possibly exorbitant price for data movement and (2) the sad fact that the RDBMS’s carefully engineered abilities to process high-volume data simply go unused.

Here, instead, we describe methods to declaratively specify complex computations using recursive SQL user-defined functions. We admit a distinctively truly functional style of function formulation — foregoing non-intuitive fixpoint semantics as well as off-putting syntactic oddities — and compile such functions into efficient pure SQL that may be executed by the RDBMS itself, right next to the data and its supporting indexes.  The compiler automatically discovers sharing and memoization opportunities during evaluation, and knows how to exploit linear as well as tail recursion .“Move you computations close to the data,” says Stonebraker and we heartily agree: RDBMSs make for capable, scalable, and declaratively programmable runtime systems.  Use them as such!
