Actors simplify distributed programming by abstracting over low-level networking and synchronization concerns, enabling greater concurrency in programs.  However, programming of concurrent systems remains difficult, among other reasons because programmers using Actors try to build ever more complex systems.  Our work attempts to address this challenge by separating a variety of concerns in Actor systems to improve modularity and reusability of code.

In two loosely related thrusts, we have worked on separating (1) location and resource concerns and (2) communication concerns.

The CyberOrgs model creates resource encapsulations for distributed Actor computations which can buy and sell resources.  The encapsulations (called cyberorgs) then become spaces of resource certainty for computations.  Cyberorgs negotiate contracts among themselves to secure resources in the future.  Resources are modeled as ticks created at the hardware-level, and then steered by processor / network / caching schedulers toward their current owners.  Cyberorgs can spawn new cyberorgs — offering autonomy to parts of their hosted computations — and can merge into other cyberorgs — giving up autonomy —  by using isolate and assimilate primitives.  

The interActors model puts communication at the center of an application, and actors carrying out computations connect to the communication at outlets and inlets.  A communication itself is made up of special-purpose actors: in addition to the outlets and inlets, there are handler actors which orchestrate complex multi-stage communications.  Although communications are intended to be programmed directly in a high-level language, they are compositionally defined in the sense that they can be constructed from primitive channels using three simple composition primitives.  This compositional definition also allows for simpler compositions to be composed to create more complex ones.

My main question to the meeting participants is: “Actors allow highly complex systems to be built, with large numbers of diverse actors and complex communications.  However, it appears that this full power is not being utilized.  Large actor systems today often have just a few types of actors interacting in simple ways.  What will it take to change this?”  
