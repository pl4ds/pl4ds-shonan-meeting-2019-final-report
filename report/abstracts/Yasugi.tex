I present a new approach to fault-tolerant language systems without a single point of failure for irregular (divide-and-conquer) parallel applications.  I propose a novel ``work omission'' paradigm and its more detailed concept as a ``hierarchical omission''-based parallel execution model called HOPE.
HOPE programmers' task is to specify which regions in imperative code can be executed in sequential but arbitrary order and how their partial results can be accessed.  HOPE workers spawn no tasks/threads at all; rather, every worker has the entire work of the sequential program with its own planned execution order, and then the workers and the underlying message mediation systems automatically exchange partial results to omit hierarchical subcomputations.  Even with fault tolerance, the HOPE framework provides parallel speedups.
As runtime systems, the message mediation systems are distributed and federated; the implementation should provide deadlock avoidance, fairness, and fault tolerance.  Unfortunately, our current implementation is in C with POSIX threads and MPI; I hope for a nice programming language to implement the complex runtime systems.
