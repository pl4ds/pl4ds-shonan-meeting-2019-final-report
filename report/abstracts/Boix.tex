%TODO%

Even though distributed systems are wide-spread, building them is still hard due to their inherent concurrency combined with the fact that components can fail independently, i.e. failures are partial.  In order to support the systematic construction of distributed systems, our research tries to aid developers with both novel programming constructs and tool support.

In this talk, I present our research on language support for replicated data. Replicating data in a distributed system can improve efficiency and availability but complicates software development because concurrent updates can lead to conflicts. In recent work (De Porre et al. 2019), we have introduced strong eventually consistent replicated objects (SECROs), a general-purpose data type for building available data structures that guarantee strong eventual consistency without requiring operations to commute. By specifying state validators arbitrary data types can be turned into highly available replicated data types. This means that replicated data types can be implemented similarly to their sequential local counterpart, with the addition of preconditions and postconditions to define concurrent semantics. We are currently working on improving the computation of valid executions to reduce runtime overhead.

To support the development of distributed applications, we study tool support in the form of debuggers. In recent work (Torres Lopez et al. 2019) presented in the demo session, we introduced multiverse debugging, a novel approach for debugging concurrent non-deterministic
programs that allows developers to observe all possible execution paths of a program and debug it interactively in a fashion similar to breakpointed-based debuggers while being probe-effect free.  This is meant to simplify the reproduction and inspection of concurrency bugs, because it removes chance and probability from the equation of hitting the problematic interleaving. Instead, an execution path that can lead to a bug can be explored interactively and a developer can see the state in all possible universes. Since  exploring the multiverses of larger programs can become unwieldy, ongoing and future research is needed to guide the exploration of the state graph, e.g. novel stepping semantics that work at the level of universes, and to develop efficient runtime techniques that make multiverse debugging practical for distributed systems.
