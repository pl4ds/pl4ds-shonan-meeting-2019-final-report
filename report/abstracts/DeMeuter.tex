Reactive programming and Distributed reactive programming is becoming a popular mainstream programming paradigm. However, even though there are many industrial strength incarnations of the idea, solid foundations are lacking. Questions such as what exactly constitutes reactive programming, how do reactive programs compose with each other and  how reactive programs interact with non-reactive programs in a predictable manner are largely unaddressed in the scientific literature.
In our research group, we are investigating a bipolar programming language design in which actors and reactors coexist with one another. Actors constitute the imperative parts of the system, i.e. they contain long lasting loops and they encapsulate the mutable state. Reactors host the parts of the system that are always(!) ready to react to events generated by third parties. Hence, reactors are always in O(1). The goal of the research is to define the distributed composition operators that allow actors and reactors to coexist in a predictable manner even in the face of partial failure.
