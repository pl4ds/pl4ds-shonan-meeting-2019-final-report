Dynamic data-driven application systems (DDDAS) allow for unprecedented self-healing and self-diagnostic behavior across a broad swathe of domains. The usefulness of these systems is offset against their inherent complexity, and therefore fragility to specification or implementation error. Further, DDDAS techniques are often applied in safety-critical domains, where correctness is paramount. Formal methods facilitate the development of correctness proofs about software systems, which provide stronger behavioral guarantees than non-exhaustive unit tests. While unit testing can validate that a system behaves correctly in some finite number of configurations, formal methods enable us to prove correctness in an infinite subset of the configuration space, which is often needed in cyber-physical systems involving continuous mechanics. Although the efficacy of formal methods is traditionally offset by significantly greater development cost, we propose new development techniques that can mitigate this concern.
					
We explore novel techniques for assuring the correctness of data-driven systems based on certified programming and software verification. In particular, we focus on the use of interactive theorem-proving systems to prove foundational properties about data-driven systems, possibly reliant upon physics-based assumptions and models. We introduce the concept of the formal safety envelope, analogous to the concept of an aircraft�s performance envelope, which organizes system properties in a way that makes it clear which properties hold under which assumptions. Beyond maintaining modularity in proof development, this technique furthermore enables the derivation of runtime monitors to detect potentially unsafe system state changes, allowing the user to know precisely which properties have been verified to hold for the current system state. Using this method, we demonstrate the partial verification of an archetypal data-driven system from avionics, where wing sensor data is used to determine whether or not an airplane is likely to be in a stall state. 
