Mobile robot systems constitute a form of distributed systems, where reaching agreement reliably is the cornerstone of cooperation. Our research focuses on these aspects, from theoretical mobile robots to swarm and multi-agent robotics. After a brief overview of our research activities, the presentation focuses on education of distributed algorithms.

Teaching distributed algorithms can be challenging and even frustratingly difficult when most students initially fail to connect the abstract models and definitions with reality. Making this link is particularly difficult because the abstract models in which distributed algorithms are expressed in textbooks are somewhat far from how such algorithms are typically implemented. In order to alleviate this problem, we have developed a simulation framework and a domain-specific language in Scala with which distributed algorithms can be expressed in a way that remains close to typical textbook pseudocode, but can also be executed in a simulated environment.
The framework, called ScalaNeko/Ocelot, inherits from the project Neko developed in Java at EPFL in 2000. ScalaNeko/Ocelot supports both active and reactive protocols, as well as their flexible composition. The network simulation allows to define arbitrary network topologies. This allows to implement, in usually less than fifty lines of code, classical distributed algorithms as varied as: detection of cut vertices, distributed mutual exclusion in arbitrary graphs, randomized leader election in anonymous networks, construction of a spanning tree, etc.
