When nodes are spread across large geographic distances, being available for local users, and providing short response times is often at odds with keeping strong consistency across the whole system. Several systems, that target large scale geo-replication, support multi-master operation and transient data divergence, allowing each site to update replicas with no immediate coordination. From the user application perspective, the system cannot be seen anymore as a single sequential copy, since now operations can be processed concurrently at different locations. Conflict-free Replicated Data Types (CRDTs) can take away a lot of the complexity when migrating from a sequential to a concurrent setting. 

Here we explore a bit of the path in this transition, cover what can be expected, and present a few guiding principles, such as: (1) permutation equivalence; (2) preserving sequential semantics; and (3) non-sequential outcomes. Evolution from sequential to concurrent behaviour was explored for common data types, such as counters, registers and sets. 
