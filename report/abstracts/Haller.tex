Distributed systems are often required to provide high scalability, reliability, and availability. We believe that programming languages play an important role in the development of distributed systems. Distributed systems are inherently concurrent, and are thus affected by a superset of hazards of concurrent software systems. Our research tries to address various hazards, including partial failure, data inconsistency, and race conditions by means of both dynamic and static approaches. In order to ensure fault tolerance, widely-used systems employ fault recovery mechanisms based on so-called lineages. A lineage maintains all information required to recover from the failure of a replica. In recent work (Haller et al. 2018), we present foundations for lineage-based distributed computation based on a programming model and type system. We prove that well-typed programs have two important properties: first, the mobility of lineages is preserved by reduction; second, materialization of remote, lineage-based data is finite. In ongoing and future work we study (a) problems related to interaction, including language constructs and type systems, and (b) problems related to latency and time. In all of these efforts, we consider modularity, scalability and availability to be essential aspects.