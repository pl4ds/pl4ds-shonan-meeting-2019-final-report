Verification of message-passing systems is a challenging task. How we structure communication can have a great influence on the difficulty of verifying message-passing programs. With a perfect network, the problem is undecidable. While considering fault makes the verification problem simpler in theory (lossy channel systems), the problem is still intractable. These results uses the communication channels to encode a large memory but hardly any real program do so. Instead, writing a program using communication-closed rounds can greatly simplify the verification. The key insight is the use of communication-closure (logical boundaries in a program that messages should not cross) to structure the code. Communication-closure gives a syntactic scope to the communication, provides some form of logical time, and give the illusion of synchrony. Communication-closed rounds are particularly suitable to implement fault-tolerant distributed algorithms. Unfortunately, communication-closed rounds have their own limitations and I will discuss our current work on lifting these limitations by borrowing ideas from multiparty session types. Session-types are more flexible and can express richer communication patterns. They also simplify the verification by integrating a top-down decomposition from the global specification to the individual components and the verification is performed at the level of individual components. On the other hand, the system models used in session types does not consider faults.